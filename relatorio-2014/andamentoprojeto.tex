%************************************************
\chapter{Andamento do projeto}\label{ch:introducao}
%************************************************
\section{Sumário do projeto inicial}
Sumário
\section{Análise do período}
Neste período, o código fonte inicial foi quase totalmente refeito para permitir maior flexibilidade. Isso foi necessário pois o código desenvolvido inicialmente e que foi utilizado em testes preliminares, abrangia apenas a predição da estruturas secundárias das proteínas e tinha como objetivo avaliar a viabilidade do projeto. 

As modificações feitas no código permitirá criar autômatos celulares com outros estados além dos já utilizados, que correspondam aos aminoácidos e a elementos de estrutura secundária. Portanto, isso possibilitará testarmos tanto códigos simplificados que representem, por exemplo, as características físico-químicas dos aminoácidos, como carga, polaridade, entre outros, assim como códigos mais complexos que representem, por exemplo, diversos elementos de estrutura secundária juntamente com a exposição ao solvente.

As disciplinas cursadas no período como  



Com a modificação dos objetivos do projeto para a criação de um método de reconhecimento de enovelamentos proteicos
\section{Discussões e conclusões parciais}
Discussões 
\section{Perspectivas futuras}
No próximo periodo
