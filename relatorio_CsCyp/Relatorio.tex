\documentclass[11pt]{report}
\usepackage[utf8]{inputenc}
\usepackage[T1]{fontenc}

\usepackage{mathpazo} % add possibly `sc` and `osf` options
\usepackage{eulervm}

\linespread{1.10}
\usepackage{indentfirst}
\setlength\parindent{1cm}

\usepackage[square]{natbib}
\usepackage{graphicx}
\usepackage[section]{placeins}



% Title Page
\title{Dinâmica Molecular CsCyp}
\author{José Geraldo de Carvalho Pereira}


\begin{document}
\maketitle

\begin{abstract}
	Colaboração com o grupo do Dr. Celso Benedetti com o objetivo de analisar a consequência estrutural/funcional da mutação S58F na proteína CsCyp. A ciclofilina CsCyp possui um mecanismo de regulação pela redução/oxidação da ponte dissulfeto C40-C168, sendo que quando esta ponte está presente, a proteína encontra-se na forma inativa. Em um trabalho anterior do grupo do Dr. Celso Benedetti sobre a CsCyp criou-se a hipótese de que a ponte dissulfeto alteraria estruturalmente o loop variável. Esta alteração seria então propagada por meio do E83 para o loop catalítico o qual fecharia o sítio ativo.
	Neste trabalho buscamos simular por meio de dinâmica molecular como a mudança S58F afetaria a atividade e também comparar os resultados com a proteína selvagem e com a mutação E83A.
\end{abstract}

%\chapter{Conhecimento pr\'evio}

\section*{Mecanismo CsCyp}

O grupo

\section*{Resultados}

\subsection*{CsCyp \textit{wt}}

 Na simulação de 100ns da CsCyp \textit{wt} e da CsCyp \textit{wt} S-S não foi observado o movimento do loop catalítico (D73-E83) descrito no mecanismo proposto pelo grupo do Dr. Celso Benedetti \citep{Campos2013}. É possível questionarmos se o tempo e o número de simulações seriam suficientes ou não para permitir observarmos essa mudança conformacional do loop catalítico fechando o sitio ativo, no entanto, há também a possibilidade de ocorrerem outras mudanças conformacionais que poderiam regular a atividade da CsCyp.
 
 Ao contrário do que poderia se esperar, a simulação sugere que as regiões do loop catalítico são estáveis em conformações muito similares durante a simulação, mas foi possível observar que outras regiões próximas ao sítio ativo sofrem efeitos tão ou até mais intensos resultantes da ligação dissulfeto $C^{40}-C^{168}$.

Uma dessas regiões é o longo loop (S88-G101) consecutivo sequencialmente ao loop catalítico. Ambos estão em contato com outro loop (A108-Q118) que também demonstra ser influenciado pela ligação S-S. Esse último, possui resíduos relacionados a atividade catalítica e a ligação do substrato como A108, N109, A110 e o G116. 

No caso da CsCyp \textit{wt} o loop (A108-Q118) apresentou maior flexibilidade em relação à CsCyp \textit{wt} S-S, sendo que nessa última ele apresentou também uma aparente mudança conformacional. Interessantemente, a flexibilidade desta região está possivelmente relacionada a atividade catalítica em CypA \citep{Eisenmesser2002} e assim, a diminuição da flexibilidade desta região estaria de acordo com a inibição da atividade nesta proteína.

Outra região que apresenta mudança conformacional e alteração na flexibilidade é a região T126-H133 a qual forma uma hélice e possui resíduos que participam do sítio ativo (W128, L129 e H133). Essa região está em contato tanto com o loop longo (S88-G101) assim como com o loop (A108-Q118) por meio da H133. Na CsCyp \textit{wt} S-S o W128 parece se distanciar do sítio ativo, o que, devido a sua função catalítica, também estaria de acordo com a inibição da proteína. Não foi possível determinar precisamente quem poderia ter maior influência na alteração estrutural da região T126-H133, mas o resíduo F95 apresentou uma diferença conformacional em sua cadeia lateral, a qual esta em contato direto com essa região, o que pode ser um indício que o loop longo (S88-G101) estaria propagando o sinal da ponte dissulfeto.



  

\begin{figure}[ht!]
	\centering
	\includegraphics[width=1.0\textwidth]{../../Pictures/cyp_noss}
	\caption{CsCyp \textit{wt}: RMSF dos últimos 50ns da simulação de 100ns}
\end{figure}
     
\begin{figure}[ht!]
	\centering
	\includegraphics[width=1.0\textwidth]{../../Pictures/cyp_ss}
	\caption{CsCyp \textit{wt} S-S: RMSF dos últimos 50ns da simulação de 100ns}
\end{figure}

\subsection*{CsCyp S58F}

As simulações de 100ns da CsCyp S58F e da CsCyp S58F S-S foram feitas para procurar compreender estruturalmente como esta mutação poderia afetar o funcionamento e a regulação da CsCyp.  

A CsCyp S58F demonstrou maior estabilidade que as CsCyp \textit{wt} na avaliação do RMSF e uma estrutura média muito similar a CsCyp \textit{wt} sem a ponte dissulfeto. Isso sugere que ela também seja a conformação ativa, mas não sabemos se a possível diferença de estabilidade está relacionada a maior ou menor eficiência enzimática.

A CsCyp S58F S-S, diferentemente da CsCyp \textit{wt} S-S, para possuir uma conformação mais semelhante a CsCyp ativa. Observamos inclusive uma maior mobilidade estrutural no loop A108-Q118 em relação a CsCyp S58F, mobilidade que, como mencionada anteriormente, pode estar relacionada a função catalítica \citep{Eisenmesser2002}. Há também uma maior estabilidade do loop catalítico em uma posição que amplia o sítio ativo, e maior mobilidade do longo loop (S88-G101) e do loop variável.

Esses dados nos permite levantar diversas hipóteses:

\begin{enumerate}
	\item A mudança S58F anula a regulação da CsCyp pela ponte dissulfeto. Uma evidência para essa hipótese seria a a semelhança entre a CsCyp S58F S-S e a CsCyp \textit{wt}.
	\item A mudança S58F poderia aumentar a eficiência da CsCyp S58F em relação a CsCyp \textit{wt}, o que seria uma consequência do aumento de estabilidade da conformação ativa.
	\item A mudança S58F poderia inverter o mecanismo de regulação. Da mesma forma que o aumento de estabilidade 
\end{enumerate}


\begin{figure}[h!]
	\centering
	\includegraphics[width=1.0\textwidth]{../../Pictures/cyp_s58f_noss}
	\caption{CsCyp S58F: RMSF dos últimos 50ns da simulação de 100ns}
\end{figure}

\begin{figure}[h!]
	\centering
	\includegraphics[width=1.0\textwidth]{../../Pictures/cyp_s58f_ss}
	\caption{CsCyp S58F S-S: RMSF dos últimos 50ns da simulação de 100ns}
\end{figure}


  
 

\bibliographystyle{cell}
\bibliography{refcscyp}

\end{document}          
